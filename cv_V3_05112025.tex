\documentclass[a4paper,10pt]{article}
\usepackage[margin=0.5in,nofoot]{geometry}
\usepackage{fontawesome5}
\usepackage{hyperref}
\usepackage{titlesec}
\usepackage{xcolor}

\hypersetup{
    colorlinks=true,
    linkcolor=black,
    filecolor=black,
    urlcolor=black,
    citecolor=black
}

\titleformat{\section}{\large\bfseries}{\thesection}{1em}{}[\titlerule]
\titlespacing*{\section}{0pt}{*1}{*1}

\newcommand{\entry}[4]{
  \noindent\textbf{#1} \hfill #2 \\
  \noindent\textit{#3} \hfill \textit{#4} \\
  \vspace{2pt}
}

\newcommand{\project}[2]{
  \noindent\textbf{#1} \hfill #2 \\
  \vspace{2pt}
}

\begin{document}

\pagenumbering{gobble}

\noindent
\begin{minipage}[t]{0.5\textwidth}
\textbf{\Large Camila Grazielle Veloso}

\vspace{0.4em}

\end{minipage}%
\begin{minipage}[t]{0.5\textwidth}
\raggedleft

Belo Horizonte - MG
\vspace{0.2em}
 \quad
{\color{blue}} \href{tel:+5531971160847}{(31)97116-0847}
{\color{blue}} \href{mailto:camila.veloso@escoladanuvem.org}{camila.veloso@escoladanuvem.org}\\

\vspace{0.2em}
 \quad
{\color{blue}} \href{https://github.com/latamila}{github.com/latamila}
{\color{blue}} \href{https://www.linkedin.com/in/camila-cientista-de-dados/}{ linkedin.com/in/camila-cientista-de-dados} \\
\end{minipage}

\vspace{1em}

\begin{center}
    \textbf{\Large Educadora em Tecnologia da Informação}
\end{center}
\vspace{0.5em}

\section*{Resumo Profissional}

\vspace{0.6em}

Educadora em Tecnologia da Informação com mais de 10 anos de experiência em Pedagogia, Psicopedagogia e áreas técnicas de TI. Cientista de dados e graduanda em Engenharia de Software, com sólida formação em linguagens de programação (Python, Java, PHP), análise de dados (SQL, Excel, Power BI) e computação em nuvem (Azure, AWS, GCP). Experiência em mentoria para certificações técnicas, com 90 porcento de aprovação em AI-900 da Azure Microsoft. Apaixonada por inovação, otimização de processos e desenvolvimento de soluções tecnológicas. Com traços, predominantemente, da personalidade arquiteto (INTJ-A), com quociente emocional de alto nível e perfil comportamental analítico, comunicativo e adaptável, com fluência em Inglês e Espanhol, busca um ambiente cultural com Obsessão no sucesso do cliente e desafios que ajudem a impulsionar o desenvolvimento da carreira na área da Educação e Tecnologia. 


\vspace{0.6em}

\vspace{0.5em}

\section*{Graduações}
\vspace{0.6em}

\entry{Bacharelado em Engenharia de Software}{2027}{}
\space
\vspace{-1.6em}

\entry{Licenciatura em Informática}{2025}{}
\space
\vspace{-1.6em}

\entry{Superior Tecnologia em Ciência de Dados}{2022}{}
\space
\vspace{-1.6em}

\entry{Licenciatura em Pedagogia}{2018}{}
\space
\vspace{-1.6em}

\entry{Licenciatura em Letras Português/Inglês}{2014}{}
\space
\vspace{-1.6em}




% \vspace{-1.6em}

\section*{Pós Graduações, Cursos e Certificações}
\vspace{0.6em}

\entry{MBA em Engenharia de Software}{2027}{630 horas}{USP}

\entry{Curso de Analista em Segurança Cibernética}{em progresso}{100 horas}{Cisco}

\entry{Certificação AI-900 Azure Microsoft}{2024}{100 horas}{Learn Microsoft}

\entry{Certificado de Engenheiro  de IA}{2024}{360 horas}{Data Science Academy}

\entry{Certificação de Fundamentos de Rede}{2024}{80 horas}{Cisco}

\entry{Curso I e II de Fundamentos da Ciência da Computação (Python)}{2023}{160 horas}{USP}

\entry{Lato Sensu em Psicologia Escolar e Educacional}{2018}{520 horas}{Universidade de Venda Nova do Imigrante}

\entry{Lato Sensu em Psicopedagogia}{2017}{360 horas}{Universidade de Uberaba}



\section*{Habilidades Técnicas}
\vspace{0.6em}
\begin{itemize}
\setlength\itemsep{0em}
\item Linguagens e Ferramentas: Python - SQL - Power BI - Excel Avançado - VSCode.
\item Banco de Dados: PostgreSQL - BigQuery - NoSQL.
\item Computação em Nuvem: Azure - AWS - Google Cloud.
\item Controle de Versão: Git e GitHub.
\item Metodologias ágeis: Scrum e Kanban.
\item Idiomas: Português - Nativo, Inglês - Avançado, Espanhol - Fluente.
\item Redes e Segurança: Modelo OSI e suas Camadas - IPv4 e IPv6 - Endereçamento DHCP - ARP - TCP - UDP - Design de Redes - Nuvem e Virtualização - Sistemas Numéricos - Comutação Ethernet - ICMP - Switches e Roteadores - Suporte de Redes - Ameaças de Segurança Cibernética, Vulnerabilidades e Ataques - Segurança de Rede - IAM

\section*{Projetos Relevantes}
\vspace{0.6em}

\entry{Escrita de artigos e documentos formais e científicos}{2025 - atual}{Escrita de artigos sobre diversas áreas de interesse em TI com a linguagem LaTex }{LinkedIn}

\entry{Professora de Carreira Tech }{2023 - atual}{Curso de Extensão para  Developer Associate da AWS}{Escola da Nuvem}

\entry{Visão Computacional para Detecção de Fraude} {2024}{Implementou um modelo de IA na Azure Machine Learning para detectar fraudes em cartões de crédito. A solução analisou padrões de transações para identificar anomalias, garantindo maior precisão e segurança. O projeto focado em visão computacional aprimorou a análise de dados visuais de cartões de crédito.
}{Github}

\entry{Mentoria Certificação AI-900}{2024}{Mentoria Individual}{Youtube Beegineers tech}

\entry{Construção da Arquitetura de Redes }{2024}{Visualizou e simulou redes no Cisco Packet Tracer assim como praticou a resolução de problemas usando exemplos autênticos. Conectou, configurou e verificou dispositivos em uma rede, incluindo a rede sem fio.}{Cisco}

\entry{Automação de envio de Mensagens no Whatsapp(CRM)}{2023}{Crédito Fácil Central}{ Github}

\section*{Habilidades Profissionais}
\vspace{0.6em}

Profissional com perfil INTJ (Arquiteto), caracterizado por pensamento estratégico, visão analítica e foco em resultados. Altamente independente e orientada a objetivos, possui habilidade para planejar e executar projetos complexos com eficiência e precisão. Demonstrando liderança natural, é capaz de identificar oportunidades de melhoria, propor soluções inovadoras e catalisar mudanças em ambientes colaborativos. Com forte capacidade de organização e autocontrole emocional, alia criatividade à lógica para resolver problemas de forma estruturada, sempre buscando excelência e otimização nos processos.

\end{itemize}

\end{document}
