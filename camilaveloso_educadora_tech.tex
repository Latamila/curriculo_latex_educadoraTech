\documentclass[a4paper,10pt]{article}
\usepackage[margin=0.5in,nofoot]{geometry}
\usepackage{fontawesome5}
\usepackage{hyperref}
\usepackage{titlesec}
\usepackage{xcolor}

\hypersetup{
    colorlinks=true,
    linkcolor=black,
    filecolor=black,
    urlcolor=black,
    citecolor=black
}

\titleformat{\section}{\large\bfseries}{\thesection}{1em}{}[\titlerule]
\titlespacing*{\section}{0pt}{*1}{*1}

\newcommand{\entry}[4]{
  \noindent\textbf{#1} \hfill #2 \\
  \noindent\textit{#3} \hfill \textit{#4} \\
  \vspace{2pt}
}

\newcommand{\project}[2]{
  \noindent\textbf{#1} \hfill #2 \\
  \vspace{2pt}
}

\begin{document}

\pagenumbering{gobble}

\noindent
\begin{minipage}[t]{0.5\textwidth}
\textbf{\Large Camila Grazielle Veloso}

\vspace{0.4em}

\end{minipage}%
\begin{minipage}[t]{0.5\textwidth}
\raggedleft

Belo Horizonte - MG

{\color{blue}} \href{tel:+5567998748431}{(31) 97116-0847}
{\color{blue}} \href{mailto:camila.veloso@escoladanuvem.org}{camila.veloso@escoladanuvem.org}

\vspace{0.2em}
 \quad
{\color{blue}} \href{https://github.com/latamila}{github.com/latamila}
{\color{blue}} \href{https://www.linkedin.com/in/camila-cientista-de-dados/}{ linkedin.com/in/camila-cientista-de-dados} \\
\end{minipage}

\vspace{1em}

\begin{center}
    \textbf{\Large Educadora em Tecnologia da Informação}
\end{center}
\vspace{0.5em}

\section*{Resumo Profissional}

\vspace{0.6em}

Instrutora de Carreiras na Instituição de Educação Profissionalizante na área de TI, a Escola da Nuvem, uma das melhores empregadores do Brasil, sem fins lucrativos. Cientista de dados e graduando-se em Engenharia de Software e Licenciatura em Informática. Mais de 10 anos de experiência diversificada em Psicopedagogia, Pedagogia e Letras Portugues/Inglês com especialidades práticas em Ciência de Dados, Engenharia de Software e Computação como as Linguagens de Programação Python, Java, PHP, HTML/CSS, Análise de Dados com Excel, SQL, as soluções de nuvem Azure, AWS e GCP e Fundamento de Redes. Ademais, lecionou mentorias para a certificação AI900 da Azure Microsoft de Inteligência Artificial , obtendo 90 porcento de seus alunos aprovados. Ama contribuir com projetos inovadores e comunidades tecnológicas, compartilhando conhecimento e experiência. Prefere um ambiente de linha de comando como Linux e MacOs. Sempre busca personalizar para encontrar o ambiente mais otimizado. Interessada em criar métodos melhores para resolver problemas e aprender novas tecnologias e ferramentas, que promovam eficiência e mais produtividade em projetos colaborativos e ágeis. 

\vspace{0.6em}

\vspace{0.5em}

\section*{Graduações}
\vspace{0.6em}

\entry{Bacharelado em Engenharia de Software}{\faCalendar \space 2027}{}
\space
\vspace{-1.6em}

\entry{Licenciatura em Informática}{\faCalendar \space 
 2025}{}
\space
\vspace{-1.6em}

\entry{Superior Tecnologia em Ciência de Dados}{\faCalendar \space 
 2022}{}
\space
\vspace{-1.6em}

\entry{Licenciatura em Pedagogia}{\faCalendar \space 
 2018}{}
\space
\vspace{-1.6em}

\entry{Licenciatura em Letras Português/Inglês}{\faCalendar \space 
 2014}{}
\space
\vspace{-1.6em}




% \vspace{-1.6em}

\section*{Pós Graduações, Cursos e Certificações}
\vspace{0.6em}

\entry{MBA em Engenharia de Software}{2027}{630 horas}{\faMapMarker \space USP}

\entry{Curso de Analista em Segurança Cibernética}{em progresso}{100 horas}{\faMapMarker \space Cisco}

\entry{Certificação AI-900 Azure Microsoft}{2024}{100 horas}{\faMapMarker \space Learn Microsoft}

\entry{Certificado de Engenheiro  de IA}{2024}{360 horas}{\faMapMarker \space Data Science Academy}

\entry{Certificação de Fundamentos de Rede}{2024}{80 horas}{\faMapMarker \space Cisco}

\entry{Curso I e II de Fundamentos da Ciência da Computação (Python)}{2023}{360 horas}{\faMapMarker \space USP}

\entry{Lato Sensu em Psicologia Escolar e Educacional}{2018}{520 horas}{\faMapMarker \space Universidade de Venda Nova do Imigrante}

\entry{Lato Sensu em Psicopedagogia}{2017}{360 horas}{\faMapMarker \space Universidade de Uberaba}



\section*{Habilidades e Competências}
\vspace{0.6em}
\begin{itemize}
\setlength\itemsep{0em}
\item Linguagens e Ferramentas: Python - SQL - Power BI - Excel Avançado - VSCode.
\item Banco de Dados: PostgreSQL - BigQuery - NoSQL.
\item Computação em Nuvem: Azure - AWS - Google Cloud.
\item Controle de Versão: Git e GitHub.
\item Metodologias ágeis: Scrum e Kanban.
\item Idiomas: Português - Nativo, Inglês - Avançado, Espanhol - Fluente.
\item Redes e Segurança: Modelo OSI e suas Camadas - IPv4 e IPv6 - Endereçamento DHCP - ARP - TCP - UDP - Design de Redes - Nuvem e Virtualização - Sistemas Numéricos - Comutação Ethernet - ICMP - Switches e Roteadores - Suporte de Redes - Ameaças de Segurança Cibernética, Vulnerabilidades e Ataques - Segurança de Rede - IAM

\section*{Experiências e Projetos Relevantes}
\vspace{0.6em}

\entry{Professora de Carreira Tech }{2023 - atual}{Curso de Extensão para  Developer Associate da AWS}{\faMapMarker \space Escola da Nuvem}

\entry{Visão da Azure para Detecção de Fraude em Cartões de Crédito }{2024}{Projeto Freelancer}{\faMapMarker \space Github}

\entry{Mentoria Certificação AI-900}{2024}{Mentoria Individualizada}{\faMapMarker \space Github}

\entry{Automação de envio de Mensagens no Whatsapp(CRM)}{2023}{Crédito Fácil Central}{\faMapMarker \space Github}

\section*{Habilidades Profissionais e Comportamentais}
\vspace{0.6em}

Profissional com perfil analítico, comunicativo, planejador e executor de maneira equilibrada. Altamente capaz de colaborar em  ambientes que prezam ou almejam a  comunicação, inspiração, gerenciamento de conflitos, influência, orientação a serviços, consciência organizacional, otimismo, iniciativa, superação, adaptabilidade, confiabilidade, autocontrole emocional, autoconfiança, organização, autonomia, empatia, compreensão, independência, discrição, agilidade, trabalho em equipe, flexibilidade, catalização da mudança, inovação,  desenvolvimento dos demais, autoavaliação precisa e autoconsciência emocional. 

\end{itemize}

\end{document}
